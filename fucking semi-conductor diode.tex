\documentclass[a4paper,14pt]{extarticle}

\def\labauthors{Сорокин Д.А., Парамонов Г.С.}
\def\labgroup{440}
\def\labnumber{2}
\def\labstartdate{Ноябрь 2022}
\def\labtheme{Измерение статических характеристик \\[0.2em] полупроводникового диода}
\def\shortlabtheme{Полупроводниковый диод}

%!TEX root = ../diode.tex
\usepackage{cmap}
\usepackage[T2A]{fontenc}
\usepackage[utf8x]{inputenc}
\usepackage[english, russian]{babel}

\usepackage{misccorr} % в заголовках появляется точка, но при ссылке на них ее нет
\usepackage{amssymb,amsfonts,amsmath,amsthm}  
\usepackage{indentfirst}
\usepackage[usenames,dvipsnames]{color} 
\usepackage[unicode,hidelinks]{hyperref}
% \hypersetup{%
%     pdfborder = {0 0 0}
% }

\usepackage{makecell,multirow} 
\usepackage{ulem}
\usepackage{graphicx,wrapfig}
\graphicspath{{img/}}
\usepackage{geometry}
\geometry{left=2cm,right=2cm,top=3cm,bottom=3cm,bindingoffset=0cm,headheight=15pt}
\usepackage{fancyhdr} 
\linespread{1.05} 
\frenchspacing 
\renewcommand{\labelenumii}{\theenumii)} 
\newcommand{\mean}[1]{\langle#1\rangle}
% \usepackage{caption}
%%%%%%%%%%%%%%%%%%%%%%%%%%%%%%%%%%%%%%%%%%%%%%%%%%%%%%%%%%%%%%%%%%%%%%%%%%%%%%%
%%%%%%%%%%%%%%%%%%%%%%%%%%%%%%%%%%%%%%%%%%%%%%%%%%%%%%%%%%%%%%%%%%%%%%%%%%%%%%%



%%%%%%%%%%%%%%%%%%%%%%%%%%%%%%%%%%%%%%%%%%%%%%%%%%%%%%%%%%%%%%%%%%%%%%%%%%%%%%%
	%применим колонтитул к стилю страницы
\pagestyle{fancy} 
	%очистим "шапку" страницы
\fancyhead{} 
	%слева сверху на четных и справа на нечетных
\fancyhead[L]{\labauthors} 
	%справа сверху на четных и слева на нечетных
\fancyhead[R]{\shortlabtheme} 
	%очистим "подвал" страницы
\fancyfoot{} 
	% номер страницы в нижнем колинтуле в центре
\fancyfoot[C]{\thepage} 
\renewcommand{\phi}{\varphi}
%%%%%%%%%%%%%%%%%%%%%%%%%%%%%%%%%%%%%%%%%%%%%%%%%%%%%%%%%%%%%%%%%%%%%%%%%%%%%%%

\usepackage{float}
\usepackage[mode=buildnew]{standalone}
\usepackage{tikz} 
% \usepackage{subcaption}
\usepackage{csvsimple}
\usetikzlibrary{scopes}
\usetikzlibrary{%
     decorations.pathreplacing,%
     decorations.pathmorphing,%
    patterns,%
    calc,%
    scopes,%
    arrows,%
    % arrows.spaced,%
}
\makeatletter
\newif\if@gather@prefix 
\preto\place@tag@gather{% 
  \if@gather@prefix\iftagsleft@ 
    \kern-\gdisplaywidth@ 
    \rlap{\gather@prefix}% 
    \kern\gdisplaywidth@ 
  \fi\fi 
} 
\appto\place@tag@gather{% 
  \if@gather@prefix\iftagsleft@\else 
    \kern-\displaywidth 
    \rlap{\gather@prefix}% 
    \kern\displaywidth 
  \fi\fi 
  \global\@gather@prefixfalse 
} 
\preto\place@tag{% 
  \if@gather@prefix\iftagsleft@ 
    \kern-\gdisplaywidth@ 
    \rlap{\gather@prefix}% 
    \kern\displaywidth@ 
  \fi\fi 
} 
\appto\place@tag{% 
  \if@gather@prefix\iftagsleft@\else 
    \kern-\displaywidth 
    \rlap{\gather@prefix}% 
    \kern\displaywidth 
  \fi\fi 
  \global\@gather@prefixfalse 
} 
\newcommand*{\beforetext}[1]{% 
  \ifmeasuring@\else
  \gdef\gather@prefix{#1}% 
  \global\@gather@prefixtrue 
  \fi
} 
\makeatother

\usepackage{booktabs}
\usepackage{pgfplots, pgfplotstable}

\usepackage[outline]{contour}
\usepackage{tocloft}
\renewcommand{\cftsecleader}{\cftdotfill{\cftdotsep}} % for parts
% \renewcommand{\cftchapleader}{\cftdotfill{\cftdotsep}} % for chapters
\usepackage{pgfplots,pgfplotstable,booktabs,colortbl}
\pgfplotsset{compat=newest}
\usepackage{physics}
\usepackage{mathtools}
\mathtoolsset{showonlyrefs=true}
\newcommand\Smat{\hat { \mathbf { S } }}

\newcommand*\dotvec[1][1,1]{\crossproducttemp#1\relax}
\def\crossproducttemp#1,#2\relax{{\qty[\vec{#1}\times\vec{#2}\,]}}

\newcommand*\prodvec[1][1,1]{\crossproducttempa#1\relax}
\def\crossproducttempa#1,#2\relax{{\qty[{#1}\times{#2}\,]}}

% \def\E{\mathscr{E}_H}
\def\Rdim{\,\frac{\text{м}^3}{\text{А} \cdot \text{с}}}

\renewcommand{\vec}{\mathbf} % for parts

\begin{document}

\begin{titlepage}
\begin{center}

{\small\textsc{Нижегородский государственный университет имени Н.\,И. Лобачевского}}
\vskip 2pt \hrule \vskip 3pt
{\small\textsc{Радиофизический факультет}}

\vfill


{{\large Отчет по лабораторной работе №\labnumber}\vskip 12pt {\LARGE \bfseries \labtheme}}

	
\vspace{2cm}
{\large Работу выполнили студенты \\[-0.25em] \labgroup\  группы радиофизического факультата \\[0.5em] {\Large \bfseries \labauthors}}


\end{center}

\vfill
	

	
\begin{center}
	{Нижний Новгород, \labstartdate\ -- \today}
\end{center}

\end{titlepage}


\tableofcontents
\newpage



\addcontentsline{toc}{section}{Введение}
\section*{Введение}
\vspace{-0.5em}
В настоящей работе изучается \textit{полупроводниковый диод}. Основной принцип работы -- инжекция основных и неосновных носителей заряда через $p-n$ переход под воздействием напряжения, приложенного к $p$ и $n$ областям. В работе будут получены экспериментальные кривые ВАХ, ВФХ, из которых будут получены константы, характеризующие свойства диода: собственная емкость, сопротивление базы, концентрация неосновных носителей заряда, коэффициент неидеальности и т.д.



\vspace{-0.5em}
 % Как показано в \cite[стр. ы]{met}

\paragraph{Установка.} В работе измеряются прямая и обратная вольт-амперная характеристика диода, вольт-фарадная характеристика контура с диодом. Измерения проводятся при комнатной температуре диода и при нагреве диода с помощью нагревательного элемента. 

Для измерения характеристик собирается лабораторная установка, общий вид которой приведен на рис. \ref{fig:1}. На диод, расположенный на корпусе над нагревательным элементом, подаётся прямое или обратное напряжение, регулируемое ручкой <<плавно>> для соответствующего напряжения. Ток и напряжение на диоде измеряются вольтметром и амперметром, выведенными на корпус установки. 

Последовательно диоду в установке включена индуктивность. За счет наличия собственной и паразитной емкости, диод и индуктивность образуют последовательный $LC$-контур, по резонансному значения тока в котором и можно получить вольт-фарадную характеристику.

\begin{figure}[H]
	\centering
	\includegraphics[width=\textwidth]{fig/view}
	\vspace{-1em}
	\caption{Лабораторная установка. 1 -- генератор сигналов UNI-T UTG9010C, 2 -- блок режимов, позволяющий подавать напряжение в прямом и обратном направлении, а также измерять резонансную частоту контура}
	\label{fig:1}
\end{figure}

\newpage

% \newpage

\section{Вольт-амперная характеристика диода}
Измерение ВАХ проводится при подключении установки, отвечающем принципиальной схеме, приведенной на рис. \ref{fig:chem} при отключенной цепи $LC$-контура. 

\begin{figure}[h!]
	\centering
	\includegraphics[width=0.75\linewidth]{fig/chem}
	\caption{Схема экспериментальной установки}
	\label{fig:chem}
\end{figure}


\subsection{ВАХ холодного диода}
Данная ВАХ снимается при условиях комнатной температуры ($\sim\hspace{-0.5em}26^\circ$). Результаты измерений обратной и прямой ветви приведены на рис. \ref{fig:vax1}, \ref{fig:vax1obr}.
\begin{figure}[H]
	\centering
	\includegraphics[scale=1.5]{fig/i_from_urev.pdf}
	\vspace{-1em}
	\caption{Обратная ветвь ВАХ}
	\label{fig:vax1obr}
\end{figure}
\begin{figure}[H]
	\centering
	\includegraphics[scale=1.5]{fig/i_from_ufor.pdf}
	\vspace{-1em}
	\caption{Прямая ветвь ВАХ}
	\label{fig:vax1}
\end{figure}
Как показано в \cite[стр. 15]{met}, ВАХ диода с коэффициентом неидеальности $n$ определяется выражением
\begin{equation}
    i=i_s[e^{(u-iR_{\text{б}})/n \varphi_T}-1].
    \label{eq:id}
\end{equation}
Для комнатной температуры $\phi_T\approx 25$ мВ. Аппроксимировав обратной функцией $u(i)$ экспериментальную ВАХ прямой ветви, получили значения неизвестных констант:
\begin{equation}
    n \simeq 1.9,\quad J_s \simeq 0.03  \text{ мА},\quad R_{\text{б}} \simeq 0.4 \text{ Ом}.
\end{equation}



\subsection{ВАХ нагретого диода}

\begin{figure}[H]
	\centering
	\includegraphics[scale=1.5]{fig/i_from_urev_heat.pdf}
	\vspace{-1em}
	\caption{Обратная ветвь ВАХ}
	\label{fig:vax2obr}
\end{figure}
\begin{figure}[H]
	\centering
	\includegraphics[scale=1.5]{fig/i_from_ufor_heat.pdf}
	\vspace{-1em}
	\caption{Прямая ветвь ВАХ}
	\label{fig:vax2}
\end{figure}
Измерения проводятся при нагреве диода с помощью нагревательного элемента. Нагрев прекращается при возрастании тока через диод до $\sim70$ мкА. 

Чтобы иметь возможность измерить характеристики при относительно постоянной температуре, на диод после нагрева надевается теплоизолирующий кожух из пенопласта.

Аналогично ВАХ прямой ветви, аппроксимацией нашли параметры кривой ВАХ:
\begin{equation}
    n \simeq 2.19,\quad J_s \simeq 0.10  \text{ мА},\quad R_{\text{б}} \simeq 0.38 \text{ Ом}.
\end{equation}
При этом считалось, что $\phi_T\approx28$ мВ ($T=340$K).


Отметим по результатам экспериментов с нагревом диода, что обратный ток нагретого диода больше, чем холодного. Это можно объяснить тем, что обратный ток возникает за счет тепловой генерации неосновных носителей заряда в нейтральных $p-$ и $n-$ областях, прилегающих к переходу. Носители диффундируют к границам перехода и переносятся в соседнюю область полем \cite[стр.14]{met}.



\newpage
\section{Вольт-фарадная характеристика диода}

Для снятия ВФХ на последовательный контур, образованный элементами $C_2=37.9$ пФ, $D$ и $L= 364$ мкГн подаётся гармонический сигнал с амплитудой 0.19 В.

Частота генератора подбирается так, чтобы наблюдать резонанс токов на частоте $f_0$. При этом частота резонанса $f_0$
\begin{equation}
	f_0 = \frac{1}{2 \pi \sqrt{L(C_{\text{д}} + C_2)}}
	\quad\Rightarrow\quad
	C_{\text{д}} = \frac{1}{4 \pi^2 L f_0^2} - C_2
	\label{eq:cd}
\end{equation}
\begin{figure}[H]
	\centering
	\includegraphics[scale=1.5]{fig/c_from_urev}
	\vspace{-1em}
	\caption{ВФХ $p-n$ перехода}
	\label{fig:cax}
\end{figure}

Как показано в \cite[стр.18]{met}, в случае резко несимметричного $p-n$ перехода величина барьерной емкости не зависит от свойств $p$-области и определяется формулой
\begin{equation}
	C_\text{д}=S\sqrt{\frac{e\varepsilon \varepsilon_oN_d}{2(U_k-U)}}
\end{equation}
Это выражение позволяет найти контактную разность потенциалов и концентрацию донорной примеси: График зависимости $~C^{-2}(U)$, изображенный на рис. \ref{fig:cax}, отсекает на оси абсцисс отрезок, равный по величине $U_k$. 

В нашем случае 
\begin{equation}
	U_k = 0.3\text{ В}.
\end{equation}
Можно также найти концентрацию донорной примеси $N_d$ через угол наклона прямой $k$:
\begin{equation}
	\frac{10}{C^2}=\frac{1}{N_d}\frac{20(U_k-U)}{e\varepsilon\varepsilon_0 S^2} \quad\Rightarrow\quad
	N_d=\frac{20}{e\varepsilon\varepsilon_0 S^2 \cdot k}
\end{equation}
В нашем случае $k=0.04 \text{ пФ}^{-2}\text{ В}^{-1}=0.04 \cdot 10^{24} \text{ Ф}^{-2}\text{ В}^{-1}$, $S^2=1\text{ мм}^2 = 10^{-6} \text{ м}^2$, $\varepsilon \sim 10$, тогда
\begin{equation}
	N_d = \frac{20}{1.6\cdot10^{-19}\cdot 8.4\cdot10^{-12}\cdot10\cdot 0.04 \cdot 10^{24}} = 3.7\cdot 10^7 \text{ м}^{-3}
\end{equation}
% \newpage
% \section{Экспериментальное ы}





\addcontentsline{toc}{section}{Заключение}
\section*{Заключение}
В настоящей работе мы изучили принципы работы полупроводникового диода, измерили основные параметры диода при комнатной температуре и в нагретом состоянии: коэффициент неидеальности $n=1.9 \divisionsymbol 2.19$, сопротивление базы $R_\text{б}=0.38\divisionsymbol0.4$ Ом, обратный ток перехода $I_s=0.03\divisionsymbol0.1$ мА, концентрацию неосновных носителей заряда $N_d=3.7\cdot 10^7 \text{ м}^{-3}$.



\begin{thebibliography}{}
  \bibitem{orlov} Орлов И.\,Я., Односевцев В.\,А. и др. Основы радиоэлектроники: учебное пособие. -- Нижний Новгород: Нижегородский государственный университет им. Н.И. Лобачевского, 2011. -- 169 с.
  
  \bibitem{met} Битюрин\,\,Ю.\,А. и др. Измерение статических характеристик полупроводникового диода. Н.Новгород: ННГУ, 2004. -- 38 с.
  
  % \bibitem{lit3} Ландау Л.Д., Лифшиц Е.М. Любой том. М.: Физматлит, 2003.
\end{thebibliography}

\end{document}
